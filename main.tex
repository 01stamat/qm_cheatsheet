% Basic stuff
\documentclass[a4paper,10pt]{article}
\usepackage[utf8]{inputenc}
\usepackage[german]{babel}
\usepackage{amsmath}
\usepackage{changepage}

% 3 column landscape layout with fewer margins
\usepackage[landscape, left=0.75cm, top=1cm, right=0.75cm, bottom=1.5cm, footskip=15pt]{geometry}
\usepackage{flowfram}
\ffvadjustfalse
\setlength{\columnsep}{1cm}
\Ncolumn{3}

% define nice looking boxes
\usepackage{tcolorbox}

% a base set, that is then customised
\tcbset {
  base/.style={
    boxrule=0mm,
    leftrule=1mm,
    left=1.75mm,
    arc=0mm, 
    fonttitle=\bfseries, 
    colbacktitle=black!10!white, 
    coltitle=black, 
    toptitle=0.75mm, 
    bottomtitle=0.25mm,
    title={#1}
  }
}

\definecolor{brandblue}{rgb}{0.34, 0.7, 1}

\newtcolorbox{mainbox}[1]{
  colframe=brandblue, 
  base={#1}
}
\newtcolorbox{credits}[1]{
  colframe=green, 
  base={#1}
}

\newtcolorbox{subbox}[1]{
  colframe=black!20!white,
  base={#1}
}
%hyperlinks
\usepackage{hyperref}
\hypersetup{
    colorlinks=true,
    linkcolor=brandblue,
    filecolor=magenta,      
    urlcolor=brandblue,
    pdftitle={Quantenmechanik Zusammenfassung}
}

\urlstyle{same}


% Mathematical typesetting & symbols
\usepackage{physics}
\usepackage{amsthm, mathtools, amssymb} 
\usepackage{marvosym, wasysym}
\usepackage{bbm}

% Tables
\usepackage{tabularx, multirow}
\usepackage{booktabs}
\renewcommand*{\arraystretch}{2}

% Make enumerations more compact
\usepackage{enumitem}
\setitemize{itemsep=0.5pt}
\setenumerate{itemsep=0.5pt}

% To include sketches
\usepackage{graphicx}

% Metadata
\title{TU Graz SS22 \break
\textbf{Quantenmechanik Zusammenfassung - Matthias Staffler}}

% Math helper stuff
\def\limn{\lim_{n\to \infty}}
\def\limx{\lim_{x\to x_0}}
\def\limxp{\lim_{x\to x_0^+}}
\def\limxm{\lim_{x\to x_0^-}^-}
\def\limxz{\lim_{x\to 0}}

\def\limxo{\lim_{x\to 0}}
\def\limxi{\lim_{x\to\infty}}
\def\limxn{\lim_{x\to-\infty}}

\def\R{\mathbb{R}}
\def\N{\mathbb{N}}
\def\C{\mathbb{C}}
\def\r{$\mathbb{R}$}
\def\n{$\mathbb{N}$}
\def\c{$\mathbb{C}$}

\newcommand{\rn}[1][n]{$\R^{#1}\ $}
\newcommand{\Rn}[1][n]{\R^{#1}}
\def\Rm{\R^m}
\def\rm{$\Rm$}
\def\dx{\text{ d}x}
\newcommand{\Mc}[1]{\mathcal{#1}}
\newcommand{\mc}[1]{$\Mc{#1}$}

\newcommand{\Xrn}[1][n]{X\subset\Rn[#1]}
\newcommand{\xrn}[1][n]{$\Xrn[#1]\ $}

\def\R{\mathbb{R}}
\def\N{\mathbb{N}}
\def\C{\mathbb{C}}
\def\dx{\text{ d}x}

\def\Pd{\partial}
\def\pd{$\Pd$}

\newcommand{\Pdo}[2]{\dfrac{\Pd #1}{\Pd #2}}
\newcommand{\pdo}[2]{$\Pdo{#1}{#2}$}

\newcommand{\sumk}[1][\infty]{\sum_{k=1}^#1}
\newcommand{\sumj}[1][\infty]{\sum_{j=1}^#1}
\newcommand{\sumn}[1][\infty]{\sum_{n=0}^#1}

\newcommand{\seq}[2][n]{$(#2_#1)_{_{#1\geq1}}$}
\newcommand{\ser}[2]{$\sum_{#1=1}^\infty #2_#1$}
\newcommand{\serz}[2]{$\sum_{#1=0}^\infty #2_#1$}

\newcommand{\Seq}[1]{(#1_n)_{_{n\geq1}}}
\newcommand{\Ser}[2]{\sum_{#1=1}^\infty #2_#1}
\newcommand{\Serz}[2]{\sum_{#1=0}^\infty #2_#1}

\newcommand{\fdr}[1][f]{$#1: D\longrightarrow\R\ $}
\newcommand{\Fdr}[1][f]{#1: D\longrightarrow\R}
\newcommand{\fct}[3]{$#1: #2\longrightarrow#3$}
\newcommand{\Fct}[3]{#1: #2\longrightarrow#3}

\newcommand{\inte}[3]{$\int_{#1}^{#2} #3(x)\ dx$}
\newcommand{\Inte}[3]{\int_{#1}^{#2} #3(x)\ dx}

\newcommand{\fndx}[2][x]{$#2^{(n)}(#1)$}
\newcommand{\Fndx}[2][x]{#2^{(n)}(#1)}
\newcommand{\fnd}[2][n]{$#2^{(#1)}$}
\newcommand{\Fnd}[2][n]{#2^{(#1)}}

\newcommand{\Fxrm}[1][m]{\Fct{f}{X}{\Rn[#1]}}
\newcommand{\fxrm}[1][m]{$\Fxrm[#1]$}

\newcommand{\str}{$(*)\ $}
\newcommand{\Str}{(*)}

\begin{document}

\section{Postulate und Formulierung}

\subsection{Eigenschaften von Kets}

Linearer Operator mit Eigenkets $\ket{a}$:
 \[ A \ket{a} = a \ket{a} \]
\section{Das freie Teilchen}
\section{Schrödinger-Gleichung}

\begin{subbox}{Zeitentwicklung}
Zeitentwicklungsoperator:
\[ \ket{a, t} = U(t, t_0) \ket{a, t_0} \]
\[ U(t_2, t_0) = U(t_2, t_1) U(t_1, t_0), \, U(t,t) = 1 \]

Hamilton-Operator:
\[ U(t_0 + dt, t_0) \ket{a, t_0} = (1-i\frac{H}{\hbar} dt ) \ket{a, t_0} + \mathcal{O}(dt^2) \]
\[ H\ket{E_n} = E_n \ket{E_n} \hspace{.8cm} \text{stationäre S-Gl.}\] 
\end{subbox}

\begin{mainbox}{Schrödinger- und Heisenbergbild}
Schrödinger-Gleichung:
\[ i\hbar \ket{a,t} = H \ket{a,t}    \]
\( H \neq H(t) \) Hamilton zeitunabhängig:
\[ \Rightarrow \ket{a, t} = \sum_n a_n(0) \, e^{-i \frac{E_n t}{\hbar}} \ket{E_n} \]
\vspace*{.5cm}
Heisenberg-Gleichung:
\[ \frac{dA}{dt} = \frac{1}{i\hbar} \comm{A}{H} \]
Polynome von $P$ und $X$:
\[ \comm{X}{F(P)} = i\hbar\,\partial_P F(P) \]
\[ \comm{P}{G(X)} = -i\hbar\,\partial_X G(X) \]
\end{mainbox}
\section{1-dimensionale Systeme}
\section{Zentralkraftproblem und Drehimpuls}
\section{Symmetrien}

\begin{subbox}{Erhaltungsgrößen}
Symmetrien sind unitäre Operatoren ($S^\dag = S^{-1}$) mit $\comm{H}{S} =0$
\[ S = 1 -i \cdot \varepsilon \frac{1}{\hbar} G + \mathcal{O}(\varepsilon^2) \]
\[ \comm{H}{G} = 0 \Rightarrow G(t) = G(0) \Rightarrow \expval{G}{t} = \expval{G}{0} \]
Erzeugte Observablen $G$ (hermitisch) sind Erhaltungsgrößen.
\end{subbox}

% \begin{subbox}{Wigner-Eckart-Theorem}
% Kotinuierliche Symmetrie $S$ mit $\comm{H}{S}=0$:
% \[H(S\ket{E}) = SH\ket{E} = ES\ket{E} \]
% \begin{enumerate}
%     \item \( S\ket{E} = \alpha \ket{E} \) mit $\alpha$ eine Phase.
%     \item \( S\ket{E} \neq \ket{E} \Rightarrow \) Degeneration 
% \end{enumerate}
% \end{subbox}

\begin{mainbox}{Parität}
\begin{align*}
    \text{ Spiegelsymmetrie: } & \vec{X} \rightarrow -\vec{X} \\
    & t \rightarrow t
\end{align*}
Operator $\Pi$ mit $\Pi^\dag X \Pi = -X$ und $\Pi^\dag = \Pi^{-1} = \Pi$ hermitisch und unitär.
\[ \Pi \ket{x} = e^{i\delta} \ket{-x}, \hspace{.2cm} \text{Konvention: } \delta = 0 \]
\[ \Pi \ket{x} = \pm \ket{x} \]
\[ \Pi \ket{l, l_3} = (-1)^l \ket{l, l_3} \]
\[ \Pi^\dag P \Pi = -P \]
Paritätseigenkets $\ket{\alpha}$ und $\ket{\beta}$ und -eigenwerte $\pi_\alpha$ und $\pi_\beta$:
\begin{align*}
\mel{\beta}{X}{\alpha} &= \mel{\beta}{\Pi^\dag \Pi X \Pi^\dag \Pi}{\alpha} = \mel{\beta}{\Pi^\dag (-X) \Pi}{\alpha} \\
 & = - \pi_\alpha \pi_\beta \mel{\beta}{X}{\alpha}
\end{align*}
Möglichkeiten:
\begin{enumerate}
    \item \(\pi_\alpha \pi_\beta = 1 \Rightarrow \mel{\beta}{X}{\alpha} = - \mel{\beta}{X}{\alpha} = 0 \)
    \item \(\pi_\alpha \pi_\beta = -1 \Rightarrow \mel{\beta}{X}{\alpha} \neq 0 \)
\end{enumerate}
\[ \expval{X}{n} = - \pi_n^2 \expval{X}{n} = - \expval{X}{n} = 0 \]
\end{mainbox}

\begin{mainbox}{Zeitumkehr}
    \begin{align*}
        \text{ Bewegungsumkehr: } & \vec{x} \rightarrow \vec{x} \\
        & \vec{P} \rightarrow -\vec{P}
    \end{align*}
Operator $\Theta$ mit:
\[ \Theta (a\ket{\alpha} + b \ket{\beta} ) = a^* \Theta \ket{\alpha} + b^* \Theta \ket{\beta} \]
\end{mainbox}
\section{Näherungsmethoden}

\end{document}
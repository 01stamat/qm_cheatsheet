% Basic stuff
\documentclass[a4paper,10pt]{article}
\usepackage[utf8]{inputenc}
\usepackage[german]{babel}
\usepackage{amsmath}
\usepackage{changepage}

% 3 column landscape layout with fewer margins
\usepackage[landscape, left=0.75cm, top=1cm, right=0.75cm, bottom=1.5cm, footskip=15pt]{geometry}
\usepackage{flowfram}
\ffvadjustfalse
\setlength{\columnsep}{1cm}
\Ncolumn{3}

% define nice looking boxes
\usepackage{tcolorbox}

% a base set, that is then customised
\tcbset {
  base/.style={
    boxrule=0mm,
    leftrule=1mm,
    left=1.75mm,
    arc=0mm, 
    fonttitle=\bfseries, 
    colbacktitle=black!10!white, 
    coltitle=black, 
    toptitle=0.75mm, 
    bottomtitle=0.25mm,
    title={#1}
  }
}

\definecolor{brandblue}{rgb}{0.34, 0.7, 1}

\newtcolorbox{mainbox}[1]{
  colframe=brandblue, 
  base={#1}
}
\newtcolorbox{credits}[1]{
  colframe=green, 
  base={#1}
}

\newtcolorbox{subbox}[1]{
  colframe=black!20!white,
  base={#1}
}
%hyperlinks
\usepackage{hyperref}
\hypersetup{
    colorlinks=true,
    linkcolor=brandblue,
    filecolor=magenta,      
    urlcolor=brandblue,
    pdftitle={Quantenmechanik Zusammenfassung}
}

\urlstyle{same}


% Mathematical typesetting & symbols
\usepackage{braket}
\usepackage{amsthm, mathtools, amssymb} 
\usepackage{marvosym, wasysym}

% Tables
\usepackage{tabularx, multirow}
\usepackage{booktabs}
\renewcommand*{\arraystretch}{2}

% Make enumerations more compact
\usepackage{enumitem}
\setitemize{itemsep=0.5pt}
\setenumerate{itemsep=0.5pt}

% To include sketches
\usepackage{graphicx}

% Metadata
\title{TU Graz SS22 \break
\textbf{Quantenmechanik Zusammenfassung - Matthias Staffler}}

% Math helper stuff
\def\limn{\lim_{n\to \infty}}
\def\limx{\lim_{x\to x_0}}
\def\limxp{\lim_{x\to x_0^+}}
\def\limxm{\lim_{x\to x_0^-}^-}
\def\limxz{\lim_{x\to 0}}

\def\limxo{\lim_{x\to 0}}
\def\limxi{\lim_{x\to\infty}}
\def\limxn{\lim_{x\to-\infty}}

\def\R{\mathbb{R}}
\def\N{\mathbb{N}}
\def\C{\mathbb{C}}
\def\r{$\mathbb{R}$}
\def\n{$\mathbb{N}$}
\def\c{$\mathbb{C}$}

\newcommand{\rn}[1][n]{$\R^{#1}\ $}
\newcommand{\Rn}[1][n]{\R^{#1}}
\def\Rm{\R^m}
\def\rm{$\Rm$}
\def\dx{\text{ d}x}
\newcommand{\Mc}[1]{\mathcal{#1}}
\newcommand{\mc}[1]{$\Mc{#1}$}

\newcommand{\Xrn}[1][n]{X\subset\Rn[#1]}
\newcommand{\xrn}[1][n]{$\Xrn[#1]\ $}

\def\R{\mathbb{R}}
\def\N{\mathbb{N}}
\def\C{\mathbb{C}}
\def\dx{\text{ d}x}

\def\Pd{\partial}
\def\pd{$\Pd$}

\newcommand{\Pdo}[2]{\dfrac{\Pd #1}{\Pd #2}}
\newcommand{\pdo}[2]{$\Pdo{#1}{#2}$}

\newcommand{\sumk}[1][\infty]{\sum_{k=1}^#1}
\newcommand{\sumj}[1][\infty]{\sum_{j=1}^#1}
\newcommand{\sumn}[1][\infty]{\sum_{n=0}^#1}

\newcommand{\seq}[2][n]{$(#2_#1)_{_{#1\geq1}}$}
\newcommand{\ser}[2]{$\sum_{#1=1}^\infty #2_#1$}
\newcommand{\serz}[2]{$\sum_{#1=0}^\infty #2_#1$}

\newcommand{\Seq}[1]{(#1_n)_{_{n\geq1}}}
\newcommand{\Ser}[2]{\sum_{#1=1}^\infty #2_#1}
\newcommand{\Serz}[2]{\sum_{#1=0}^\infty #2_#1}

\newcommand{\fdr}[1][f]{$#1: D\longrightarrow\R\ $}
\newcommand{\Fdr}[1][f]{#1: D\longrightarrow\R}
\newcommand{\fct}[3]{$#1: #2\longrightarrow#3$}
\newcommand{\Fct}[3]{#1: #2\longrightarrow#3}

\newcommand{\inte}[3]{$\int_{#1}^{#2} #3(x)\ dx$}
\newcommand{\Inte}[3]{\int_{#1}^{#2} #3(x)\ dx}

\newcommand{\fndx}[2][x]{$#2^{(n)}(#1)$}
\newcommand{\Fndx}[2][x]{#2^{(n)}(#1)}
\newcommand{\fnd}[2][n]{$#2^{(#1)}$}
\newcommand{\Fnd}[2][n]{#2^{(#1)}}

\newcommand{\Fxrm}[1][m]{\Fct{f}{X}{\Rn[#1]}}
\newcommand{\fxrm}[1][m]{$\Fxrm[#1]$}

\newcommand{\str}{$(*)\ $}
\newcommand{\Str}{(*)}

\newcommand{\Pmat}[1]{\begin{pmatrix} #1 \end{pmatrix}}
\newcommand{\Bmat}[1]{\begin{bmatrix} #1 \end{bmatrix}}
\newcommand{\Mat}[1]{\begin{matrix} #1 \end{matrix}}
\newcommand{\pmat}[1]{$\begin{pmatrix} #1 \end{pmatrix}$}
\newcommand{\bmat}[1]{$\begin{bmatrix} #1 \end{bmatrix}$}
\newcommand{\mat}[1]{$\Mat{#1}$}

\DeclareMathOperator{\Ker}{Ker}
\DeclareMathOperator{\Dim}{Dim}
\DeclareMathOperator{\Tr}{Tr}

\begin{document}

\section{Postulate und Formulierung}

\begin{mainbox}{Eigenschaften von Kets}
Linearer Operator mit Eigenkets $\ket{a}$:
 \[ A \ket{a} = a \ket{a} \]
Erwartungswert:
 \[ \expval{A}_s = \mel{s}{A}{s} \hspace*{1cm}, \hspace{1cm} \expval{A}_a = a \braket{a}{a} = a \]
Flukuation und Dispersion:
\[ \Delta A = A - \expval{A} \]
\[ \expval{(\Delta A)^2} = \expval{A^2} - \expval{A}^2 \]
Unschärferelation:
\[ \expval{(\Delta A)^2}\cdot \expval{(\Delta B)^2} \geq \frac{1}{4} | \expval{\comm{A}{B}}|^2 \]
\end{mainbox}


\section{Das freie Teilchen}

\begin{subbox}{Ort und Impuls}
Ortsoperator:
\[ X \ket{x} = x \ket{x} \]
Translationsoperator:
\[ T(dx) \ket{x} = \ket{x + dx}\]
\[ T(dx) = 1 - i G dx + \mathcal{O}(dx^2) \]
\[ T(dx) = 1 - \frac{i}{\hbar} P dx + \mathcal{O}(dx^2) \Longleftrightarrow G = \frac{P}{\hbar} \] %fix 1 for 1 operator
Kommutator:
\[ \comm{X_i}{P_j} = i\hbar \delta_{ij} \]
\end{subbox}

\begin{subbox}{Wellenfunktionen}
Kontinuumsformulierung:
\[ \braket{a}{b} = \int dx  \braket{a}{x} \braket{x}{b} = \int dx \,\psi_a^*(x)\psi_b(x) \]
Erwartungswert für $X$:
\[ \ev{X}{a} = \int dx \, x \cdot \psi_a^*(x) \psi_a(x) = \int dx \, x |\psi_a(x)|^2 \]
Matrixelement für $P$:
\[ \mel**{x}{\frac{iP}{\hbar}}{a} = \frac{\partial \psi_a(x)}{\partial x} \]
\end{subbox}

\begin{mainbox}{Darstellungen}
Ortsdarstellung: 
\[ (\ket{a}, X, P) \longleftrightarrow (\psi_a(x), x, -i\hbar\partial_x) \]
Impulsdarstellung:
\[ (\ket{a}, P, X) \longleftrightarrow (\phi_a(p), p, i\hbar\partial_p) \]

Darstellungswechsel:
\[ \mel{x}{P}{p} =  p \braket{x}{p} \]
\[ \mel{x}{P}{p} = -i\hbar \partial_x \braket{x}{p} \]
\[ \Rightarrow \braket{x}{p} = \psi_p(x) = C e^{\frac{ipx}{\hbar}} \]
\[ \psi_a(x) = \braket{x}{a} = \int dp \, \braket{x}{p} \braket{p}{a} = \int \, \frac{dp}{\sqrt{2\pi \hbar}} e^{\frac{ipx}{\hbar}} \phi_a(p) \]

\end{mainbox}
\section{Schrödinger-Gleichung}

\begin{subbox}{Zeitentwicklung}
Zeitentwicklungsoperator:
\[ \ket{a, t} = U(t, t_0) \ket{a, t_0} \]
\[ U(t_2, t_0) = U(t_2, t_1) U(t_1, t_0), \, U(t,t) = 1 \]

Hamilton-Operator:
\[ U(t_0 + dt, t_0) \ket{a, t_0} = (1-i\frac{H}{\hbar} dt ) \ket{a, t_0} + \mathcal{O}(dt^2) \]
\end{subbox}

\begin{mainbox}{Schrödinger-Gleichung}
\[ i\hbar \ket{a,t} = H \ket{a,t}    \]
\end{mainbox}
\section{1-dimensionale Systeme}

\begin{subbox}{Lösungsstrategie}
\begin{enumerate}
    \item Asymptotische Analyse
    \item Reihenansatz
    \item Reihenabbruchbedingung suchen
\end{enumerate}
\end{subbox}

\begin{mainbox}{Harmonischer Oszillator}
\[ H = \frac{P^2}{2m} + \frac{m\omega^2}{2} X^2 \]
Schrödinger-Gleichung lösen:
\begin{align*}
    \Rightarrow E_n  &= \hbar \omega (n + \frac{1}{2}) \\
    \Rightarrow \psi_n(x, t) &= \frac{1}{\sqrt{a_H}} \frac{\pi^{-\frac{1}{4}}}{\sqrt{2^n n!}} \cdot H_n \left(\frac{x}{a_H}\right) e^{-\frac{x^2}{2a_H^2}-i\frac{E_nt}{\hbar}}
\end{align*}
Operatorsprache:
\[ a = \sqrt{\frac{m \omega}{2\hbar}} \left(X + \frac{iP}{m\omega}\right), \hspace{0.6cm} a^\dag = \sqrt{\frac{m \omega}{2\hbar}} \left(X - \frac{iP}{m\omega}\right) \]
\[ N = a^\dag a \]
\[\Rightarrow H = \hbar \omega \left(N + \frac{1}{2}\right), \hspace{.6cm} N \ket{E_n} = n \ket{E_n} \]
\end{mainbox}

\begin{subbox}{Erzeuger- und Vernichter}
\begin{align*}
\comm*{a}{a^\dag} &= 1 \\
\comm*{N}{a} &= -a \\
\comm*{N}{a^\dagger} &= a^\dagger
\end{align*}
Wirkung von $a$ und $a^\dagger$: 
\[ a\ket{n} = \sqrt{n} \ket{n-1} \]
\[ a^\dagger{n} = \sqrt{n+1} \ket{n+1} \]
Ort und Impuls ausgedrückt mit Erzeuger und Vernichter:
\[ X = \sqrt{\frac{\hbar}{2m\omega}} (a + a^\dagger) \]
\[ P = \frac{1}{i} \sqrt{\frac{m\omega\hbar}{2}} (a-a^\dagger) \]
\end{subbox}
\section{Zentralkraftproblem und Drehimpuls}

\begin{mainbox}{Drehimpuls}
\[ \vec{L} = \vec{X} \cross \vec{P} \]
\[\expval{L_3} = \hbar l_3\]
Drehimpulsalgebra:
\[\comm{L_i}{L_j} = i\hbar \sum_k \varepsilon_{ijk} L_k \]
\[L_3 \ket{l, l_3} = \hbar l_3 \ket{l, l_3} \]
\[\vec{L}^2 \ket{l, l} = \hbar^2 l(l+1) \ket{l, l} \]
\end{mainbox}

\begin{mainbox}{Coulomb-Potential}
\[ V(r) = \frac{\gamma}{r} \]
Stationäre Schrödinger-Gleichung lösen:
\begin{align*}
    E =& -\frac{l^2}{2a_B} \frac{Z}{n^2} \\
    \Psi_{nll_3} =& \frac{2}{n^2}\sqrt{\frac{Z^3}{a_B^3}}\sqrt{\frac{(n-1-l)!}{(n+l)!}} \left(\frac{2Zr}{na_B}\right)^l \\ & \cdot L_{n-l-1}^{2l+1}\left(\frac{2Zr}{na_B}\right) e^{-\frac{Zr}{na_B}}Y_{ll_3}(\theta, \omega)
\end{align*}
Quantenzahlen für Zustand $\ket{n, l, l_3}$:
\begin{align*}
    n: &\text{ Hauptquantenzahl} &\\
    l: &\text{ Bahndrehimpuls} &\\
    & l \leq n-1 &\\
    l_3: &\text{ z-Komponente Bahndrehimpuls. } &\\
    & l_3 = {-l, \dots , 0, \dots , l} &\\
    &(2l+1)\text{-fach entartet.}
\end{align*}
\end{mainbox}

\begin{mainbox}{Harmonischer Oszillator in 3D}
Kartesische Koordinaten:
\[ H = \sum_i \left( -\frac{\hbar^2}{2m}\partial_i ^2 + \frac{m\omega_i^2}{2}x_i^2\right) \]
\begin{align*}
    \Rightarrow & E = \sum_i \hbar \omega_i \left(n_i + \frac{1}{2}\right) \\
    & H = \sum_i \hbar \omega_i \left(a_i^\dag a_i + \frac{1}{2}\right)
\end{align*}

Kugelkoordinaten und Drehimpuls:
\begin{align*}
    \Rightarrow E =& 2n + l + \frac{3}{2} \\
     \phi_{nll_3} =& \sqrt{\frac{2n!}{(n+l+\frac{1}{2})!}} \frac{r^l}{a_h^{l+\frac{3}{2}}} L_n^{l+\frac{1}{2}} \left(\frac{r}{a_h}\right) e^{-\frac{r^2}{2a_h^2}} Y_{ll_3} (\theta, \omega)
\end{align*}
\end{mainbox}
\section{Symmetrien}

\begin{subbox}{Erhaltungsgrößen}
Symmetrien sind unitäre Operatoren ($S^\dag = S^{-1}$) mit $\comm{H}{S} =0$
\[ S = 1 -i \cdot \varepsilon \frac{1}{\hbar} G + \mathcal{O}(\varepsilon^2) \]
\[ \comm{H}{G} = 0 \Rightarrow G(t) = G(0) \Rightarrow \expval{G}{t} = \expval{G}{0} \]
Erzeugte Observablen $G$ (hermitisch) sind Erhaltungsgrößen.
\end{subbox}

% \begin{subbox}{Wigner-Eckart-Theorem}
% Kotinuierliche Symmetrie $S$ mit $\comm{H}{S}=0$:
% \[H(S\ket{E}) = SH\ket{E} = ES\ket{E} \]
% \begin{enumerate}
%     \item \( S\ket{E} = \alpha \ket{E} \) mit $\alpha$ eine Phase.
%     \item \( S\ket{E} \neq \ket{E} \Rightarrow \) Degeneration 
% \end{enumerate}
% \end{subbox}

\begin{mainbox}{Parität}
\begin{align*}
    \text{ Spiegelsymmetrie: } & \vec{X} \rightarrow -\vec{X} \\
    & t \rightarrow t
\end{align*}
Operator $\Pi$ mit $\Pi^\dag X \Pi = -X$ und $\Pi^\dag = \Pi^{-1} = \Pi$ hermitisch und unitär.
\[ \Pi \ket{x} = e^{i\delta} \ket{-x}, \hspace{.2cm} \text{Konvention: } \delta = 0 \]
\[ \Pi \ket{x} = \pm \ket{x} \]
\[ \Pi \ket{l, l_3} = (-1)^l \ket{l, l_3} \]
\[ \Pi^\dag P \Pi = -P \]
Paritätseigenkets $\ket{\alpha}$ und $\ket{\beta}$ und -eigenwerte $\pi_\alpha$ und $\pi_\beta$:
\begin{align*}
\mel{\beta}{X}{\alpha} &= \mel{\beta}{\Pi^\dag \Pi X \Pi^\dag \Pi}{\alpha} = \mel{\beta}{\Pi^\dag (-X) \Pi}{\alpha} \\
 & = - \pi_\alpha \pi_\beta \mel{\beta}{X}{\alpha}
\end{align*}
Möglichkeiten:
\begin{enumerate}
    \item \(\pi_\alpha \pi_\beta = 1 \Rightarrow \mel{\beta}{X}{\alpha} = - \mel{\beta}{X}{\alpha} = 0 \)
    \item \(\pi_\alpha \pi_\beta = -1 \Rightarrow \mel{\beta}{X}{\alpha} \neq 0 \)
\end{enumerate}
\[ \expval{X}{n} = - \pi_n^2 \expval{X}{n} = - \expval{X}{n} = 0 \]
\end{mainbox}

\begin{mainbox}{Zeitumkehr}
    \begin{align*}
        \text{ Bewegungsumkehr: } & \vec{x} \rightarrow \vec{x} \\
        & \vec{P} \rightarrow -\vec{P}
    \end{align*}
Operator $\Theta$ mit:
\[ \Theta (a\ket{\alpha} + b \ket{\beta} ) = a^* \Theta \ket{\alpha} + b^* \Theta \ket{\beta} \]
\end{mainbox}
\section{Näherungsmethoden}

\begin{mainbox}{Störungstheorie ohne Degeneration}
Hamilton-Operator wird in bekannten $H_0$ und Störpotenzial $V$ aufgeteilt.
\[ H = H_0 + \lambda V \]
Energiedifferenz zu ungestörten Energieniveaus:
\[\Delta_n = E_n - E_n^{(0)} \]
Erste Ordnung in Störungstheorie:
\[\Delta_n^{(1)} = V_{nn} = \mel{n^{(0)}}{V}{k^{(0)}} \]
\[\ket*{n^{(1)}} = \sum_{n^{(0)}\neq k^{(0)}} \frac{V_{kn}}{E_n^{(0)}-E_k^{(0)}} \ket*{k^{(0)}} \]
Zweite Ordnung in Störungstheorie:
\[ \Delta_n^{(2)} = \sum_{n^{(0)}\neq k^{(0)}} \frac{|V_{kn}|^2}{E_n^{(0)}-E_k^{(0)}} \]
\begin{align*}
 \ket*{n^{(2)}} =& \sum_{n\neq k} \sum_{l\neq n} \frac{V_{kl}V_{ln}}{\left(E_n^{(0)}-E_k^{(0)}\right)\left(E_n^{(0)} - E_l^{(0)}\right)} \ket*{k^{(0)}}  \\ 
 & -\sum_{n\neq k} \frac{V_{nn} V_{kn}}{\left(E_n^{(0)}-E_k^{(0)}\right)^2} \ket*{k^{(0)}}
\end{align*}
Näherung:
\[ \ket*{n} = \ket*{n^{(0)}} + \lambda \ket*{n^{(1)}} + \lambda ^2 \ket*{n^{(2)}} + \dots \]
\[ \Delta_n = \lambda \Delta_n^{(1)} + \lambda^2 \Delta_n^{(2)} + \dots \]
\end{mainbox}

% \begin{subbox}{Degenerierte Störungstheorie}
    
% \end{subbox}

\begin{subbox}{Variationsansatz}
    
\end{subbox}

\end{document}
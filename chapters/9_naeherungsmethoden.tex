\section{Näherungsmethoden}

\begin{mainbox}{Störungstheorie ohne Degeneration}
Hamilton-Operator wird in bekannten $H_0$ und Störpotenzial $V$ aufgeteilt.
\[ H = H_0 + \lambda V \]
Energiedifferenz zu ungestörten Energieniveaus:
\[\Delta_n = E_n - E_n^{(0)} \]
Erste Ordnung in Störungstheorie:
\[\Delta_n^{(1)} = V_{nn} = \mel{n^{(0)}}{V}{k^{(0)}} \]
\[\ket*{n^{(1)}} = \sum_{n^{(0)}\neq k^{(0)}} \frac{V_{kn}}{E_n^{(0)}-E_k^{(0)}} \ket*{k^{(0)}} \]
Zweite Ordnung in Störungstheorie:
\[ \Delta_n^{(2)} = \sum_{n^{(0)}\neq k^{(0)}} \frac{|V_{kn}|^2}{E_n^{(0)}-E_k^{(0)}} \]
\begin{align*}
 \ket*{n^{(2)}} =& \sum_{n\neq k} \sum_{l\neq n} \frac{V_{kl}V_{ln}}{\left(E_n^{(0)}-E_k^{(0)}\right)\left(E_n^{(0)} - E_l^{(0)}\right)} \ket*{k^{(0)}}  \\ 
 & -\sum_{n\neq k} \frac{V_{nn} V_{kn}}{\left(E_n^{(0)}-E_k^{(0)}\right)^2} \ket*{k^{(0)}}
\end{align*}
Näherung:
\[ \ket*{n} = \ket*{n^{(0)}} + \lambda \ket*{n^{(1)}} + \lambda ^2 \ket*{n^{(2)}} + \dots \]
\[ \Delta_n = \lambda \Delta_n^{(1)} + \lambda^2 \Delta_n^{(2)} + \dots \]
\end{mainbox}

% \begin{subbox}{Degenerierte Störungstheorie}
    
% \end{subbox}

\begin{subbox}{Variationsansatz}
    
\end{subbox}